%%  It's a sample file!
%%  Please use corressponding template

\documentclass[gic]{ipart}

\RequirePackage{amsthm,amsmath,amssymb}
\RequirePackage{hyperref}

\startlocaldefs
%\numberwithin{equation}{section}
\theoremstyle{plain}
\newtheorem{thm}{Theorem}[section]
\def\SM#1#2{\sum_{#1\in #2}}
\def\FL#1{\left\lfloor #1 \right\rfloor}
\def\FR#1#2{{\frac{#1}{#2}}}
\endlocaldefs

%%  Settings
\pubyear{2019}
\volume{0}
\issue{0}
\firstpage{1}
\lastpage{9}
\arxiv{0000.0000}

\begin{document}

\begin{frontmatter}
\title[A sample document]{A sample document\protect\thanksref{T1}}
\thankstext{T1}{Footnote to the title with the `thankstext' command.}

\begin{aug}
    \author{\fnms{First} \snm{Author}\thanksref{t2}\ead[label=e1]{first@somewhere.com}},
    \address{Address of the First Author\\
             Country\\
             \printead{e1}}
    \author{\fnms{Second} \snm{Author}\ead[label=e2]{second@somewhere.com}},
    \address{Address of the Second Author\\
             Country\\
             \printead{e2}}
    \and
    \author{\fnms{Third} \snm{Author}
            \ead[label=e3]{third@somewhere.com}%
            \ead[label=u1,url]{http://www.foo.com}}
    \address{Address of the Third Author\\
             Country\\
             \printead{e3}\\
             \printead{u1}}
    \thankstext{t2}{Footnote to the author with the `thankstext' command.}
\end{aug}
%
\received{\sday{3} \smonth{1} \syear{2019}}

\begin{abstract}
The abstract should summarize the contents of the paper.
It should be clear, descriptive, self-explanatory and not longer
than 200 words. It should also be suitable for publication in
abstracting services. Please avoid using math formulas as much as possible.

This is a sample input file.  Comparing it with the output it
generates can show you how to produce a simple document of
your own.
\end{abstract}

\begin{keyword}[class=AMS]
\kwd[Primary ]{00K00}
\kwd{00K01}
\kwd[; secondary ]{00K02}
\end{keyword}

%%  Upper case for every keyword
\begin{keyword}
\kwd{Sample}
\kwd{\LaTeXe}
\end{keyword}

%\tableofcontents
\end{frontmatter}

\section{Ordinary text}

The ends  of words and sentences are marked
  by   spaces. It  doesn't matter how many
spaces    you type; one is as good as 100.  The
end of   a line counts as a space.

One   or more   blank lines denote the  end
of  a paragraph.

Since any number of consecutive spaces are treated like a single
one, the formatting of the input file makes no difference to
      \TeX,
but it makes a difference to you.
When you use
      \LaTeX,
making your input file as easy to read as possible
will be a great help as you write your document and when you
change it.  This sample file shows how you can add comments to
your own input file.

Because printing is different from typewriting, there are a
number of things that you have to do differently when preparing
an input file than if you were just typing the document directly.
Quotation marks like       ``this''
have to be handled specially, as do quotes within quotes:
       ``\,`this'          % \, separates the double and single quote.
    is what I just
    wrote, not  `that'\,''.

Dashes come in three sizes: an
       intra-word
dash, a medium dash for number ranges like
       1--2,
and a punctuation
       dash---like
this.

A sentence-ending space should be larger than the space between words
within a sentence.  You sometimes have to type special commands in
conjunction with punctuation characters to get this right, as in the
following sentence.
       Gnats, gnus, etc.\    % `\ ' makes an inter-word space.
       all begin with G\@.   % \@ marks end-of-sentence punctuation.
You should check the spaces after periods when reading your output to
make sure you haven't forgotten any special cases.
Generating an ellipsis
       \ldots\    % `\ ' needed because TeX ignores spaces after
                  % command names like \ldots made from \ + letters.
%
% Note how a `%' character causes TeX to ignore the
% end of the input line, so these blank lines do not
% start a new paragraph.
with the right spacing around the periods
requires a special  command.

\TeX\ interprets some common characters as commands, so you must type
special commands to generate them.  These characters include the
following:
       \& \% \# \{ and~\}.

In printing, text is emphasized by using an        
{\em italic\/}    % The \/ command produces the tiny extra space that
                  % should be added between a slanted and a following
                  % unslanted letter.
type style.

\begin{em}
   A long segment of text can also be emphasized in this way.  Text within
   such a segment given additional emphasis
      with\/ {\em Roman}
   type.  Italic type loses its ability to emphasize and become simply
   distracting when used excessively.
\end{em}

It is sometimes necessary to prevent \TeX\ from breaking a line where
it might otherwise do so.  This may be at a space, as between the
``Mr.'' and ``Jones'' in
``Mr.~Jones'',    % ~ produces an unbreakable interword space.
or within a word---especially when the word is a symbol like
       \mbox{\em itemnum\/}
that makes little sense when hyphenated across
       lines.

\TeX\ is good at typesetting mathematical formulas like
       \( x-3y = 7 \)
or
       \( a_{1} > x^{2n} / y^{2n} > x' \).
Remember that a letter like
       $x$    % $ ... $  and  \( ... \)  are equivalent
is a formula when it denotes a mathematical symbol, and should
be treated as one.

\begin{table*}
\tabcolsep=0pt
\caption{The spherical case ($I_1=0$, $I_2=0$)}\label{sphericcase}
\begin{tabular*}{\textwidth}{@{\extracolsep{\fill}}crrrrc}
\hline
 Equil. Points & \multicolumn{1}{c}{$x$} & \multicolumn{1}{c}{$y$} & \multicolumn{1}{c}{$z$} & \multicolumn{1}{c}{$C$} &
S \\
\hline
$L_1$ & $-$2.485252241 & 0.000000000 & 0.017100631 & 8.230711648 & U \\
$L_2$ &    0.000000000 & 0.000000000 & 3.068883732 & 0.000000000 & S \\
$L_3$ &    0.009869059 & 0.000000000 & 4.756386544 & $-$0.000057922 & U \\
$L_4$ &    0.210589855 & 0.000000000 & $-$0.007021459 & 9.440510897 & U \\
$L_5$ &    0.455926604 & 0.000000000 & $-$0.212446624 & 7.586126667 & U \\
$L_6$ &    0.667031314 & 0.000000000 & 0.529879957 & 3.497660052 & U \\
$L_7$ &    2.164386674 & 0.000000000 & $-$0.169308438 & 6.866562449 & U \\
$L_8$ &    0.560414471 & 0.421735658 & $-$0.093667445 & 9.241525367 & U \\
$L_9$ &    0.560414471 & $-$0.421735658 & $-$0.093667445 & 9.241525367 & U\\
$L_{10}$ & 1.472523232 & 1.393484549 & $-$0.083801333 & 6.733436505 & U \\
$L_{11}$ & 1.472523232 & $-$1.393484549 & $-$0.083801333 & 6.733436505 & U
\\ \hline
\end{tabular*}
\end{table*}


\section{Notes}
Footnotes\footnote{This is an example of a footnote.}
pose no problem\footnote{And another one.}.

\section{Displayed text}

Text is displayed by indenting it from the left margin.
Quotations are commonly displayed.  There are short quotations
\begin{quote}
   This is a short a quotation.  It consists of a
   single paragraph of text.  There is no paragraph
   indentation.
\end{quote}
and longer ones.
\begin{quotation}
   This is a longer quotation.  It consists of two paragraphs
   of text.  The beginning of each paragraph is indicated
   by an extra indentation.

   This is the second paragraph of the quotation.  It is just
   as dull as the first paragraph.
\end{quotation}
Another frequently-displayed structure is a list.
The following is an example of an {\em itemized} list, four levels deep.
\begin{itemize}
\item  This is the first item of an itemized list.  Each item
      in the list is marked with a ``tick''.  The document
      style determines what kind of tick mark is used.
\item  This is the second item of the list.  It contains another
      list nested inside it.  The three inner lists are an {\em itemized}
      list.
    \begin{itemize}
       \item This is the first item of an enumerated list that
            is nested within the itemized list.
          \item This is the second item of the inner list.  \LaTeX\
            allows you to nest lists deeper than you really should.
      \end{itemize}
      This is the rest of the second item of the outer list.  It
      is no more interesting than any other part of the item.
   \item  This is the third item of the list.
\end{itemize}


The following is an example of an {\em enumerated} list, four levels deep.
\begin{enumerate}
\item  This is the first item of an enumerated list.  Each item
      in the list is marked with a ``tick''.  The document
      style determines what kind of tick mark is used.
\item  This is the second item of the list.  It contains another
      list nested inside it.  The three inner lists are an {\em enumerated}
      list.
    \begin{enumerate}
       \item This is the first item of an enumerated list that
            is nested within the enumerated list.
          \item This is the second item of the inner list.  \LaTeX\
            allows you to nest lists deeper than you really should.
      \end{enumerate}
      This is the rest of the second item of the outer list.  It
      is no more interesting than any other part of the item.
   \item  This is the third item of the list.
\end{enumerate}


The following is an example of a {\em description} list.
\begin{description}
\item[Cow] Highly intelligent animal that can produce milk out of grass.
\item[Horse] Less intelligent animal renowned for its legs.
\item[Human being] Not so intelligent animal that thinks that it can think.
\end{description}

You can even display poetry.
\begin{verse}
   There is an environment for verse \\    % The \\ command separates lines
   Whose features some poets will curse.   % within a stanza.

               % One or more blank lines separate stanzas.

   For instead of making\\
   Them do {\em all\/} line breaking, \\
   It allows them to put too many words on a line when they'd
   rather be forced to be terse.
\end{verse}

Mathematical formulas may also be displayed.  A displayed formula is
one-line long; multiline formulas require special formatting
instructions
   \[  x' + y^{2} = z_{i}^{2}.\]
Don't start a paragraph with a displayed equation, nor make
one a paragraph by itself.

Example of a theorem:


\begin{thm}
All conjectures are interesting, but some conjectures are more
interesting than others.
\end{thm}

\begin{proof}
Obvious.
\end{proof}

\section{Tables and figures}
Cross references to labelled tables you can see in Table~\ref{sphericcase}
and also in Table~\ref{parset}.

A major
point of difference lies in the value of the specific production rate $\pi$ for
large values of the specific growth rate $\mu$.
Already in the early publications \cite{r1,r2,r3}
it appeared that high glucose
concentrations in the production phase are well correlated with a
low penicillin yield (the
`glucose effect'). It has been confirmed recently
\cite{r1,r2,r3,r4}
that
high glucose concentrations inhibit the synthesis of the enzymes of the
penicillin pathway, but not the actual penicillin biosynthesis.
In other words, glucose represses (and not inhibits) the penicillin
biosynthesis.

These findings do not contradict the results of
\cite{r5} and of \cite{r6} which were obtained for
continuous culture fermentations.
Because for high values of the specific
growth rate $\mu$ it is most likely (as shall be discussed below) that
maintenance metabolism occurs, it can be shown that
in steady state continuous culture conditions, and with $\mu$ described by a Monod kinetics
\begin{equation}
    C_{s}  =  K_{M} \frac{\mu/\mu_{x}}{1-\mu/\mu_{x}} \label{cs}
\end{equation}
Pirt and Rhigelato determined $\pi$ for $\mu$ between
$0.023$ and $0.086$ h$^{-1}$.
They also reported a value $\mu_{x} \approx 0.095$
h$^{-1}$, so that for their experiments $\mu/\mu_{x}$ is in the range
of $0.24$ to $0.9$.
Substituting $K _M$ in Eq. (\ref{cs}) by
the value $K_{M}=1$ g/L as used by \cite{r1}, one finds
with the above equation $0.3 < C_{s} < 9$ g/L. This agrees well with
the work of \cite{r7}, who reported that penicillin biosynthesis
repression only occurs at glucose concentrations from $C_{s}=10$ g/L on.
The conclusion is that the glucose concentrations in the experiments of
Pirt and Rhigelato probably were too low for glucose repression to be
detected. The experimental data published by Ryu and Hospodka
are not detailed sufficiently to permit a similar analysis.

\begin{table}
\centering
\caption{Parameter sets used by Bajpai and Reu\ss are not detailed sufficiently to permit a similar analysis}
\label{parset}
\begin{tabular}{lrll}
\hline
\multicolumn{2}{@{}l}{Parameter} & Set 1 & Set 2\\
\hline
$\mu_{x}$           & [h$^{-1}$]  & 0.092       & 0.11          \\
$K_{x}$             & [g/g DM]     & 0.15        & 0.006         \\
$\mu_{p}$           & [g/g DM h]  & 0.005       & 0.004         \\
$K_{p}$             & [g/L]        & 0.0002      & 0.0001        \\
$K_{i}$             & [g/L]        & 0.1         & 0.1           \\
$Y_{x/s}$           & [g DM/g]     & 0.45        & 0.47          \\
$Y_{p/s}$           & [g/g]        & 0.9         & 1.2           \\
$k_{h}$             & [h$^{-1}$]  & 0.04        & 0.01          \\
$m_{s}$             & [g/g DM h]  & 0.014       & 0.029         \\
\hline
\end{tabular}
\end{table}

Bajpai and Reu\ss\ decided to disregard the
differences between time constants for the two regulation mechanisms
(glucose repression or inhibition) because of the
relatively very long fermentation times, and therefore proposed a Haldane
expression for $\pi$.

It is interesting that simulations with the \cite{r4} model for the
initial conditions given by these authors indicate that, when the
remaining substrate is fed at a constant rate, a considerable and
unrealistic amount of penicillin is
produced when the glucose concentration is still very high \cite{r2,r3,r4}.
Simulations with the Bajpai and Reu\ss\ model correctly predict almost
no penicillin production in similar conditions.

\begin{figure}
\centering
\fbox{\makebox[5cm][c]{Picture}\rule[-1.5cm]{0pt}{3cm}}
%\includegraphics{}
\caption{Pathway of the penicillin G biosynthesis.}\label{penG}
\end{figure}

Sample of cross-reference to figure.
Figure~\ref{penG} shows that is not easy to get something on paper.


\section{Headings}

\subsection{Subsection}
Carr-Goldstein based their model on balancing methods and
biochemical know\-ledge. The original model (1980) contained an equation for the
oxygen dynamics which has been omitted in a second paper
(1981). This simplified model shall be discussed here.

\subsubsection{Subsubsection}
Carr-Goldstein
based their model on balancing\break methods and
biochemical know\-ledge. The original model (1980) contained an equation for the
oxygen dynamics which has been omitted in a second paper
(1981). This simplified model shall be discussed here.

\section{Equations and the like}

Two equations:
\begin{equation}
    C_{s}  =  K_{M} \frac{\mu/\mu_{x}}{1-\mu/\mu_{x}} \label{ccs}
\end{equation}
and
\begin{equation}
    G = \frac{P_{\rm opt} - P_{\rm ref}}{P_{\rm ref}} \mbox{\ }100 \mbox{\ }(\%).
\end{equation}

Two equation arrays:
\begin{align}
  \frac{dS}{dt} & =  - \sigma X + s_{F} F\\
  \frac{dX}{dt} & =    \mu    X\\
  \frac{dP}{dt} & =    \pi    X - k_{h} P\\
  \frac{dV}{dt} & =    F
\end{align}
and
\begin{align}
 \mu_{\rm substr} & =  \mu_{x} \frac{C_{s}}{K_{x}C_{x}+C_{s}}  \\
 \mu              & =  \mu_{\rm substr} - Y_{x/s}(1-H(C_{s}))(m_{s}+\pi /Y_{p/s}) \\
 \sigma           & =  \mu_{\rm substr}/Y_{x/s}+ H(C_{s}) (m_{s}+ \pi /Y_{p/s})
\end{align}

Long equation:
\begin{align}
\SM u{C^+}\FL{\FR{w'(u)}m}&\le \FL{\SM u{C^+} \FR{w'(u)}m}\nonumber\\ &\le
\FL{\FR{r(v)+\SM u{C^+} w(u)}m} = \FL{\FR{w(v)-\SM u{C^-} w(u)}m}\nonumber\\
&\le \FL{\FR{w(v)}m}-\SM u{C^-}\FL{\FR{w(u)}m}
= s(v)+\SM u{C^+}\FL{\FR{w(u)}m}
\end{align}
and
\begin{align}
\SM u{C^-}\FL{\FR{w(u)}m}&\le \FL{\SM u{C^-} \FR{w(u)}m}\nonumber\\ & \le
\FL{\FR{r'(v)+\SM u{C^-} w'(u)}m} = \FL{\FR{w'(v)-\SM u{C^+} w'(u)}m}\nonumber\\
&\le \FL{\FR{w'(v)}m}-\SM u{C^+}\FL{\FR{w'(u)}m}
= s'(v)+\SM u{C^-}\FL{\FR{w'(u)}m}.
\end{align}
This time we have
\begin{align*}
f(S)-f(T) =    {}  &  D_k^T(1+C_{\geq k+1}^T)(1 + C) + C_k^T(1+D_{\geq k+1}^T)(1 + D) \\
        {}   &  - C_k^T(1+C_{\geq k+1}^T)(1 + C) - D_k^T(1+D_{\geq k+1}^T)(1 + D)   \\
=     {} &   (D_k^T-C_k^T)[(1+C_{\geq k+1})(1+C)-(1+D_{\geq
k+1}^T)(1+D)]>0.
\end{align*}

\appendix

\section{Appendix section}\label{app}

We consider a sequence of queueing systems
indexed by $n$.  It is assumed that each system
is composed of $J$ stations, indexed by $1$
through $J$, and $K$ customer classes, indexed
by $1$ through $K$.  Each customer class
has a fixed route through the network of
stations.  Customers in class
$k$, $k=1,\ldots,K$, arrive to the
system according to a
renewal process, independently of the arrivals
of the other customer classes.  These customers
move through the network, never visiting a station
more than once, until they eventually exit
the system.

\subsection{Appendix subsection}

However, different customer classes may visit
stations in different orders; the system
is not necessarily ``feed-forward.''
We define the {\em path of class $k$ customers} in
as the sequence of servers
they encounter along their way through the network
and denote it by
\begin{equation}
\mathcal{P}=\bigl(j_{k,1},j_{k,2},\dots,j_{k,m(k)}\bigr). \label{path}
\end{equation}

Sample of cross-reference to the formula \ref{path} in Appendix~\ref{app}.

\section*{Acknowledgements}
And this is an acknowledgements section with a heading that was produced by the
\verb|\section*| command. Thank you all for helping me writing this
\LaTeX\ sample file.

\begin{thebibliography}{9}

\bibitem{r1} C. A. Athanasiadis, On a refinement of the generalized
Catalan numbers for Weyl groups. \textit{Trans. AMS} \textbf{357} (2005), 179--196.
\MR{2098091}

\bibitem{r2} D. Bessis, The dual braid monoid,
\textit{Ann. Sci. \'Ecole Norm. Sup.} S\'er. IV \textbf{36} (2003), 647--683.
\MR{2032983}

\bibitem{r3} N. Bourbaki,
\textit{Groupes et alg\`ebres de Lie}, Chapitres 4, 5 et 6, Hermann,
Paris (1975).
\MR{0453824}

\bibitem{r4} P. Cellini and P. Papi,
ad-nilpotent ideals of a Borel subalgebra. \textit{J.~Algebra} \textbf{225} (2000), 130--141.
\MR{1743654}

\bibitem{r5} P. Cellini and P. Papi,
ad-nilpotent ideals of a Borel subalgebra II. \textit{J.~Algebra} \textbf{258} (2002), 112--121.
\MR{1958899}

\bibitem{r6}
R. P. Dilworth,  Proof of a conjecture on finite modular
lattices.
\textit{Annals Math.} \textbf{60} (1954), 359--364.
\MR{0063348}

\bibitem{r7}
J. E. Humphreys,  \textit{Reflection Groups and Coxeter Groups}.
Cambridge Univ. Press. (1992).
\MR{1066460}

\end{thebibliography}

\end{document}
